%% Copernicus Publications Manuscript Preparation Template for LaTeX Submissions
%DIF LATEXDIFF DIFFERENCE FILE
%DIF DEL text/submission/main.tex   Mon Aug  7 09:18:32 2023
%DIF ADD text/revision/main.tex     Fri Aug 11 15:37:47 2023
%% ---------------------------------
%% This template should be used for copernicus.cls
%DIF 4c4-5
%DIF < %% The class file and some style files are bundled in the Copernicus Latex Package, which can be downloaded from the different journal webpages.
%DIF -------
%% The class file and some style files are bundled in the Copernicus Latex Package, which can be  %DIF > 
%% downloaded from the different journal webpages. %DIF > 
%DIF -------
%% For further assistance please contact Copernicus Publications at: production@copernicus.org
%% https://publications.copernicus.org/for_authors/manuscript_preparation.html


%DIF 9c10-12
%DIF < %% Please use the following documentclass and journal abbreviations for preprints and final revised papers.
%DIF -------
%% Please use the following documentclass and journal abbreviations for preprints and final  %DIF > 
%% revised  %DIF > 
%% papers. %DIF > 
%DIF -------

%% 2-column papers and preprints
\documentclass[gmd, manuscript]{copernicus}



%% Journal abbreviations (please use the same for preprints and final revised papers)


% Advances in Geosciences (adgeo)
% Advances in Radio Science (ars)
% Advances in Science and Research (asr)
% Advances in Statistical Climatology, Meteorology and Oceanography (ascmo)
% Annales Geophysicae (angeo)
% Archives Animal Breeding (aab)
% Atmospheric Chemistry and Physics (acp)
% Atmospheric Measurement Techniques (amt)
% Biogeosciences (bg)
% Climate of the Past (cp)
% DEUQUA Special Publications (deuquasp)
% Drinking Water Engineering and Science (dwes)
% Earth Surface Dynamics (esurf)
% Earth System Dynamics (esd)
% Earth System Science Data (essd)
% E&G Quaternary Science Journal (egqsj)
%DIF 35c38-39
%DIF < % EGUsphere (egusphere) | This is only for EGUsphere preprints submitted without relation to an EGU journal.
%DIF -------
% EGUsphere (egusphere) | This is only for EGUsphere preprints submitted without relation to an  %DIF > 
% EGU journal. %DIF > 
%DIF -------
% European Journal of Mineralogy (ejm)
% Fossil Record (fr)
% Geochronology (gchron)
% Geographica Helvetica (gh)
% Geoscience Communication (gc)
% Geoscientific Instrumentation, Methods and Data Systems (gi)
% Geoscientific Model Development (gmd)
% History of Geo- and Space Sciences (hgss)
% Hydrology and Earth System Sciences (hess)
% Journal of Bone and Joint Infection (jbji)
% Journal of Micropalaeontology (jm)
% Journal of Sensors and Sensor Systems (jsss)
% Magnetic Resonance (mr)
% Mechanical Sciences (ms)
% Natural Hazards and Earth System Sciences (nhess)
% Nonlinear Processes in Geophysics (npg)
% Ocean Science (os)
% Polarforschung - Journal of the German Society for Polar Research (polf)
% Primate Biology (pb)
% Proceedings of the International Association of Hydrological Sciences (piahs)
% Safety of Nuclear Waste Disposal (sand)
% Scientific Drilling (sd)
% SOIL (soil)
% Solid Earth (se)
% The Cryosphere (tc)
% Weather and Climate Dynamics (wcd)
% Web Ecology (we)
% Wind Energy Science (wes)


%% \usepackage commands included in the copernicus.cls:
%\usepackage[german, english]{babel}
%\usepackage{tabularx}
%\usepackage{cancel}
%\usepackage{multirow}
%\usepackage{supertabular}
%\usepackage{algorithmic}
%\usepackage{algorithm}
%\usepackage{amsthm}
%\usepackage{float}
%\usepackage{subfig}
%\usepackage{rotating}
%DIF PREAMBLE EXTENSION ADDED BY LATEXDIFF
%DIF UNDERLINE PREAMBLE %DIF PREAMBLE
\RequirePackage[normalem]{ulem} %DIF PREAMBLE
\RequirePackage{color}\definecolor{RED}{rgb}{1,0,0}\definecolor{BLUE}{rgb}{0,0,1} %DIF PREAMBLE
\providecommand{\DIFadd}[1]{{\protect\color{blue}\uwave{#1}}} %DIF PREAMBLE
\providecommand{\DIFdel}[1]{{\protect\color{red}\sout{#1}}}                      %DIF PREAMBLE
%DIF SAFE PREAMBLE %DIF PREAMBLE
\providecommand{\DIFaddbegin}{} %DIF PREAMBLE
\providecommand{\DIFaddend}{} %DIF PREAMBLE
\providecommand{\DIFdelbegin}{} %DIF PREAMBLE
\providecommand{\DIFdelend}{} %DIF PREAMBLE
\providecommand{\DIFmodbegin}{} %DIF PREAMBLE
\providecommand{\DIFmodend}{} %DIF PREAMBLE
%DIF FLOATSAFE PREAMBLE %DIF PREAMBLE
\providecommand{\DIFaddFL}[1]{\DIFadd{#1}} %DIF PREAMBLE
\providecommand{\DIFdelFL}[1]{\DIFdel{#1}} %DIF PREAMBLE
\providecommand{\DIFaddbeginFL}{} %DIF PREAMBLE
\providecommand{\DIFaddendFL}{} %DIF PREAMBLE
\providecommand{\DIFdelbeginFL}{} %DIF PREAMBLE
\providecommand{\DIFdelendFL}{} %DIF PREAMBLE
%DIF COLORLISTINGS PREAMBLE %DIF PREAMBLE
\RequirePackage{listings} %DIF PREAMBLE
\RequirePackage{color} %DIF PREAMBLE
\lstdefinelanguage{DIFcode}{ %DIF PREAMBLE
%DIF DIFCODE_UNDERLINE %DIF PREAMBLE
  moredelim=[il][\color{red}\sout]{\%DIF\ <\ }, %DIF PREAMBLE
  moredelim=[il][\color{blue}\uwave]{\%DIF\ >\ } %DIF PREAMBLE
} %DIF PREAMBLE
\lstdefinestyle{DIFverbatimstyle}{ %DIF PREAMBLE
	language=DIFcode, %DIF PREAMBLE
	basicstyle=\ttfamily, %DIF PREAMBLE
	columns=fullflexible, %DIF PREAMBLE
	keepspaces=true %DIF PREAMBLE
} %DIF PREAMBLE
\lstnewenvironment{DIFverbatim}{\lstset{style=DIFverbatimstyle}}{} %DIF PREAMBLE
\lstnewenvironment{DIFverbatim*}{\lstset{style=DIFverbatimstyle,showspaces=true}}{} %DIF PREAMBLE
%DIF END PREAMBLE EXTENSION ADDED BY LATEXDIFF

\begin{document}

\title{Getting the leaves right matters for estimating temperature extremes}

% \Author[affil]{given_name}{surname}

\author[1,2]{Gregory Duveiller}%
\author[3]{Mark Pickering}%
\author[4]{Joaquin Muñoz-Sabater}%
\author[2]{Luca Caporaso}%
\author[4]{Souhail Boussetta}%
\author[4]{Gianpaolo Balsamo}%
\author[2]{Alessandro Cescatti}%

\affil[1]{Max Planck Institute for Biogeochemistry, Jena, Germany}
\affil[2]{European Commission Joint Research Centre, Ispra, Italy}
\affil[3]{JRC consultant, Ispra, Italy}
\affil[4]{European Centre for Medium Range Weather Forecasts, Reading, UK}


%DIF < % The [] brackets identify the author with the corresponding affiliation. 1, 2, 3, etc. should be inserted.
%DIF > % The [] brackets identify the author with the corresponding affiliation. 1, 2, 3, etc. should 
%DIF > % be inserted.

%DIF < % If an author is deceased, please mark the respective author name(s) with a dagger, e.g. "\Author[2,$\dag$]{Anton}{Smith}", and add a further "\affil[$\dag$]{deceased, 1 July 2019}".
%DIF > % If an author is deceased, please mark the respective author name(s) with a dagger, e.g. 
%DIF > % "\Author[2,$\dag$]{Anton}{Smith}", 
%DIF > % and add a further "\affil[$\dag$]{deceased, 1 July 2019}".

%DIF < % If authors contributed equally, please mark the respective author names with an asterisk, e.g. "\Author[2,*]{Anton}{Smith}" and "\Author[3,*]{Bradley}{Miller}" and add a further affiliation: "\affil[*]{These authors contributed equally to this work.}".
%DIF > % If authors contributed equally, please mark the respective author names with an asterisk, 
%DIF > % e.g. 
%DIF > % "\Author[2,*]{Anton}{Smith}" and "\Author[3,*]{Bradley}{Miller}" and add a further 
%DIF > % affiliation: 
%DIF > % "\affil[*]{These authors contributed equally to this work.}".


\correspondence{Gregory Duveiller (gduveiller@bgc-jena.mpg.de)}

\runningtitle{TEXT}

\runningauthor{TEXT}





\received{}
\pubdiscuss{} %% only important for two-stage journals
\revised{}
\accepted{}
\published{}

%% These dates will be inserted by Copernicus Publications during the typesetting process.


\firstpage{1}

\maketitle



\begin{abstract}

Atmospheric reanalyses combine observations and models through data assimilation techniques to 
provide spatio-temporally continuous fields of key surface variables. They can do so for extended 
historical periods whilst ensuring a coherent representation of the main Earth system cycles. 
ERA5, and its enhanced land surface component ERA5-Land, are widely used in Earth System science 
and form the flagship products of the Copernicus Climate Change Service (C3S) of the European 
Commission. Such land surface modelling frameworks generally rely on a state variable called leaf 
area index (LAI), representing the amount of leaves in a grid cell at a given time, to quantify 
the fluxes of carbon, water and energy between the vegetation and the atmosphere. However, the 
LAI within the modelling framework behind ERA5 and ERA5-Land is prescribed as a climatological 
seasonal cycle, neglecting any inter-annual variability and the potential consequences that this 
uncoupling between vegetation and atmosphere may have on the surface energy balance and the 
climate. To evaluate the impact of this mismatch in LAI, we analyse the corresponding effect it has on 
land surface temperature (LST) by comparing what is simulated to satellite observations. We 
characterise a hysteretic behaviour between LST biases and LAI biases that evolves differently along 
the year depending on the background climate. We further analyse their repercussion on the 
reconstructed climate during the more extreme conditions in terms of LAI deviations, with a specific 
focus on the 2003, 2010 and 2018 heatwaves in Europe where LST mismatches are exacerbated. We 
anticipate that our results will assist users of ERA5 and ERA5-Land data to understand where and 
when the larger discrepancies can be expected, but also guide developers towards improving the 
modelling framework. Finally, this study could provide a blueprint for a wider benchmarking 
framework for land surface model evaluation that exploits the capacity of LST to integrate the effects 
of both radiative and non-radiative processes affecting the surface energy.
\end{abstract}


%\copyrightstatement{TEXT} %% This section is optional and can be used for copyright transfers.


\introduction  %% \introduction[modified heading if necessary]


The state of the land surface modulates the exchange of water and energy
between the land and the atmosphere~\citep{Seneviratne_2010}. It can thus
affect the physical state of the atmosphere and therefore influence the
seasonal to inter-seasonal predictability and climate
projections~\citep{Koster_2004}.~The biophysical land-atmosphere
interactions are determined by land surface properties, such as albedo,
emissivity, surface roughness and evaporation~\citep{anderson2012climate}, all of
which can be highly heterogeneous in both space and
time~\citep{santanello2018land}. As a result, the partition of~available energy
into latent and sensible heat fluxes can be highly variable over emerged
surfaces of the planet~\citep{dickinson1995land}. The result of this allocation
has a direct impact on local surface or near-surface air
temperature~\citep{pielke2002influence}, which in turn can exacerbate the impacts
of anthropogenic climate change.~

The type and density of vegetation covering the land surface have a
strong role in determining the surface energy balance. Land cover is
normally classified into broad groups summarising land surface
properties involved in land-climate interactions. Under similar
conditions of radiation, a forest will generally absorb more energy than
low vegetation (i.e. grasses or crops) due to its darker surface but, in
terms of surface temperature, this is generally more than compensated by
the larger amount of energy released back to the atmosphere through
higher transpiration, which itself is possible due to improved access to
water through deeper roots~\citep{Bonan2008}. Differences in land cover
have been shown to affect land surface temperature
(LST)~\citep{Duveiller_2018, Alkama2016, Li2015} and even affect the cloud regime above
them~\citep{Duveiller_2021, Xu2022}. However, land surface characteristics also vary
at different time-scales within similar land cover classes and are
further affected by both natural processes and land
management~\citep{anderson2011biophysical}.~Particularly in extra-tropical regions,
land characteristics exhibit strong seasonal patterns due to the cycle
of leaf development and senescence, influencing the seasonality of
albedo, surface roughness length and fluxes of water
and energy~\citep{RICHARDSON2013156}. Another way to characterise the overall
state of the vegetated land that more readily catches such differences
in a spatially continuous way is to consider the state variable known as
Leaf Area Index (LAI).
\DIFaddbegin 

\DIFaddend LAI is defined as half of the total green leaf area
per unit horizontal ground surface area~\citep{yan2019review}. \DIFdelbegin \DIFdel{It is a key
variable as it represents }\DIFdelend \DIFaddbegin \DIFadd{The reason for only considering 
half of the total area in this definition (rather than simply the one-sided leaf area) is to ensure 
non-flat leaves to be considered according to the actual surface area, which is proportional to their capacity to exchange water and carbon.
LAI's importance is indeed that it can represent }\DIFaddend the exchange surface between plants and the atmosphere, at 
the intersection between water, energy and carbon cycles, thus playing a critical role in the 
feedback of vegetation to the climate system~\citep{fang2019overview, Forzieri_2017}. 
LAI exhibits a large seasonal variability accordingly to climate zones and vegetation types and a substantial 
inter-annual variability linked to year-to-year variability
in weather or in management~\citep{boussetta2015assimilation}. To a large extent,
LAI drives the temporal changes in biophysical properties within a given
land cover type, since some properties, such as albedo and stomatal
conductance, can still differ among vegetation types for the same value
of LAI due for instance to morphological differences in leaf types (e.g.
broadleaf versus needleleaf). In any case, in any effort to estimate or
monitor land-atmosphere interactions and their consequences, getting the
quantities of leaves right seems to be an important consideration for
climate reconstruction and prediction.~

The impacts of land-atmosphere interactions on local temperature are
exacerbated during extreme events such as
heatwaves~\citep{jia2019land}.~The trigger of such events is often an
atmospheric circulation anomaly governed by persistent
anticyclones~\citep{schubert2014northern, brunner2018}, enabling cloud-free conditions and an 
increase 
of net solar radiation.~ Heatwaves can be locally intensified
by land-atmosphere feedbacks, which in turn may result in enhanced
growth of the atmospheric boundary layer that increases the entrainment
of heat~\citep{Miralles_2011} and/or horizontal heat
advection~\citep{schumacher2019amplification}. In addition, a deficit in the soil
moisture content can further warm the air~\citep{hauser2016role}~ so that
both thermodynamic and dynamic drivers could act
synergically~\citep{coumou2018influence}, leading to an amplification of major
heatwaves~\citep{horton2016review}. As a result, heatwaves often occur
as compound events characterised by a persistent drought that increases
the intensity of the heatwave~\citep[e.g.][]{miralles2012soil,Seneviratne_2010}. Studies
suggest that dense vegetation can limit the amplitude of heat
extremes~\citep{renaud2009comparison}, with deciduous and mixed forests having a
stronger cooling effect~compared to conifer forests. Understanding the
role of vegetation states on these phenomena is becoming increasingly
relevant as heatwaves~have increased in intensity, frequency and
duration~\citep{perkins2020increasing}, with these trends getting worse as the
climate warms up~\citep{christidis2015,coumou2018influence}~due to various factors such as the
increased climate variability~\citep{schar2004role}, a weakening of soil
moisture constraints~\citep{Rasmijn_2018} and reduced plant transpiration
due to CO\textsubscript{2} physiological forcing~\citep{skinner2018amplificationa}.~In
addition, observation data reveal a stronger increase of high
temperatures over land compared to trends in global mean temperature,
and this is particularly true for the most~extreme
events~\citep{seneviratne2014no}. The relevance and impact of land atmosphere
interactions is also likely to extend to more northern regions, as
demonstrated in a recent study on heatwaves over northern
Europe~\citep{Dirmeyer_2021}.

To monitor the changing state of the Earth System, including heatwaves,
it is essential to have reliable data that is spatially and temporally
consistent and modelling frames that mechanistically represent the
interplay between the key variables. Although the availability of Earth
Observation (EO) data has been increasing in terms of quality, quantity
and diversity, they remain constrained by two main issues: (1) EO
records can have spatio-temporal gaps and (2) several state variables
can simply not be measured directly. These shortcomings can be
compensated by integrating observations within a modelling framework,
which is where {reanalysis }comes into play.~ By optimally combining
observations and models through data assimilation techniques, reanalyses
can provide spatio-temporally continuous fields of variables for an
extended historical period while ensuring the integrity and coherence in
the representation of the main Earth system cycles~\citep{Hersbach_2020, Dee_2011}.

One of the most widely-used reanalysis for Earth System Science is the
atmospherical reanalysis of the European Centre for Medium-Range
Weather Forecasts (ECMWF). Currently, the latest installment of this
dataset is the fifth generation of atmospheric reanalysis
called~ERA5~\citep{Hersbach_2020}, and it is produced using 4D-Var
data~assimilation and~a ECMWF~model forecast~(the Integrated Forecast
System (IFS))~version corresponding to the ECMWF cycle cy41r2. Within
the IFS, an atmospheric model is coupled both to an ocean model and to a
land-surface model, the latter being responsible for correctly
representing the land-atmosphere interactions introduced above. This
model was originally called TESSEL, for Tiled ECMWF Scheme for Surface
Exchanges over Land. It was revised to address shortcomings of the land
surface scheme to represent the hydrology to become
HTESSEL~\citep{Balsamo_2009}. An additional land surface CO2 exchange
module was added to enable environmental forecasting applications which
also involves interaction with atmospheric CO2 concentration, leading to
CHTESSEL~\citep{Boussetta_2013}. The land surface model has more recently
evolved into ECLand, a modular system that should facilitate modular
extensions for the benefit of efficient developments and external
collaborations~\citep{Boussetta_2021}.

ERA5 is now a flagship product of the European Commission's Copernicus
Climate Change Service (C3S)~{and is widely used across diverse fields}.
Within C3S, ECMWF has also produced an enhanced land component of
ERA5, known as ERA5-Land~\citep{Munoz_Sabater_2021}.~It is produced by
re-running the land component of the ERA5 reanalysis at a finer spatial
resolution driven by the original atmospheric forcing from ERA5. This
results in land variables at a higher horizontal resolution
(\(\approx\)~9~km) than those available from ERA5
(\(\approx\)~31~km).~ It is also cost-effective way to produce
very consistent land variables over several decades, as observations are
not directly assimilated and the land component is not coupled to the
atmospheric or ocean model. The fact that both ERA5 and ERA5-Land are
now an integral and operational part of C3S means that their production
is guaranteed with timely updates.~Furthermore, following the efforts of
the CO2 Human Emissions (CHE) project~\citep{Balsamo_2021}, ECLand should
become the engine of the prototype Copernicus CO2 monitoring tool within
the follow-up CoCO2 project (\url{https://coco2-project.eu/}). Given its
prominent role in all these initiatives,~there is a high interest in
further evaluation the capacity of ECLand within the ERA5 and ERA5-Land
framework to correctly represent land-atmosphere interactions, in
particular under the extreme conditions of heatwaves.

{In order to correctly characterise land-atmosphere interactions, a
variable that a modelling system should ideally predict accurately is
LST, as it governs the interface between water and energy
fluxes.~}Several studies have revealed how the land model behind ERA5
and ERA5-Land suffers from a strong bias~in its representation of
LST~\citep{Johannsen_2019,Nogueira_2020,Orth_2017}. They all conclude that incorrect 
descriptions
of 
the vegetation are largely responsible for such poor model
performances.~\citet{Orth_2017} demonstrated that there is no region
across Europe or Africa where both mean LST or its seasonal dynamics are
well captured by the CHTESSEL model, but they also suggest that
considerable improvement can be gained by calibrating with multiple
observation-driven datasets. Focussing on the Iberian
Peninsula,~\citet{Johannsen_2019} found that replacing the land cover
representation with a newer ESA-CCI map could reduce the summer
bias.~\citet{Nogueira_2020} confirmed that this LST bias problem with
CHTESSEL was also present in the widely used ERA5 data. They further
showed how another land surface model (SURFEX-ISBA) did not display the
cold bias over Iberia and attributed this improvement to both a better
land cover description and a more appropriate seasonal evolution of
LAI, including a clumping parametrization for low vegetation. Based on
these results,~\citet{Nogueira_2021} updated both land cover and
vegetation seasonality in the ECMWF coupled system to show the potential
of reducing the LST bias beyond Iberia. This work however highlights
the complex regional heterogeneity in the atmospheric sensitivity to
land cover and vegetation changes, calling for re-calibration of the
model parameters and re-evaluation of model assumptions for future
reanalyses.

Although the misrepresentation of vegetation types has clearly been
identified as a main culprit in the shortcomings of LST representation
in ERA5 and ERA5-Land, there is still a main issue that has yet to be
investigated: LAI dynamics. In both ERA5 and ERA5-Land, LAI is always
prescribed at grid cell level with an identical seasonal cycle based on
satellite-derived LAI~\citep{Boussetta_2012}. While this had been
considered as an improvement compared to a more basic look-up table
approach employed in the past~\citep{Boussetta_2012}, it still neglects any
inter-annual variability in the phenology and density of vegetation.
This means that any year where a variation is observed from this
climatological seasonal cycle, in terms of either phase or amplitude,
will lead to a discrepancy between reality and the land representation
in the modelling framework. Such mismatch may be particularly
exacerbated in situations of heatwaves, as plant phenology has been
shown to vary substantially under dry/hot extremes and have important
impacts on the development of these events~\citep{stefanon2012effects, 
skinner2018amplificationa, 
Lorenz_2013}. The benchmarking exercises mentioned 
before~\citep{Johannsen_2019,Nogueira_2020,Nogueira_2021} 
only considered changes in the sources of the LAI products, but always kept
it as a prescribed seasonal cycle, leaving no room to explore the
dynamical nature of the LST bias with respect to the LAI
mismatch.

In this paper, we propose an alternative take on evaluating the
importance of LAI variations on the LST biases within the modelling
framework that produces the ERA5 and ERA5-Land datasets.~We focus
specifically on the dynamic nature of these biases and their possible
repercussion on the accuracy in the reconstruction of heatwaves. The
objective of this work is two-fold: (1) to make a comprehensive
diagnostic of how the combined biases in LAI and LST evolve in space and
time and across climate zones; (2) to evaluate the repercussion this has
on our capacity to represent heatwaves over Europe and set the basis for
a future improvement of the system.


\section{Material and methods}


\subsection{Reanalysis data}

The main data~used~in this study is the reanalysis data. All of it is
available from the Copernicus Data Store (CDS) of the C3S service
(\url{https://cds.climate.copernicus.eu/}). The priority is to
investigate the data in ERA5-Land, as users interested in land and
land-atmosphere interactions would probably opt for the dedicated land
product at the finer spatial resolution of 0.1\selectlanguage{ngerman}\textbf{°} rather than
ERA5. Besides, ERA5-Land~shows very good consistency~in the longer time
records, whereas ERA5 surface variables with long memory present
frequent inconsistencies~\citep{Munoz_Sabater_2021}. However, some variables
needed for the study are only available in ERA5. Therefore, the entire
study is focused on the 0.25\textbf{°} grid of ERA5, and all variables
used, whether from ERA5-Land or from satellite products, are aggregated
back to the 0.25\textbf{°} grid. To avoid any confusion and to remind
the reader that the underlying data is mostly pertinent to ERA5-Land, we
will henceforth use the acronym ERA5L to refer to the dataset prepared
in this study, while reserving ERA5 and ERA5-Land to design the original
data sources. The time period considered for ERA5L ranges from~{2003
until 2018}.

Each variable from reanalysis needs to be matched with a respective
equivalent from satellite-derived products that serves as an
`observational' reference.~ For most of these variables, there are
generally various different sources of satellite products to choose
from. The choices we made were guided by the aim to use products that
are independent from the ERA5, ERA5-Land and C3S environments. For each
variable, the match between the satellite reference and the reanalysis
variables requires some specific considerations, and these will be
discussed on a per-variable basis in the following subsections.


\subsection{Leaf Area Index}

The satellite-derived LAI product that we use in this study is
GEOV2/AVHRR (Verger et al. 2021)\DIFaddbegin \DIFadd{, which is available at~}\url{https://www.theia-land.fr/en/product/series-of-vegetation-variables-avhrr/}\DIFaddend . This product is based on applying a
neural network retrieval algorithm on the AVHRR Long Term Data Record
(LTDR, version 4, available
at~\url{https://ltdr.modaps.eosdis.nasa.gov/cgi-bin/ltdr/ltdrPage.cgi}).
Additionally, this product benefits from a thorough pre-retrieval
spectral harmonization and a post-retrieval gap-filling procedure. This
product was designed to have a high consistency with the GEOV2-CGLS
products derived from VEGETATION and PROBA-V sensors, distributed by
the~Copernicus~Global Land Service (CGLS), and which have been found to
improve the LST bias in previous studies~\citep{Nogueira_2020,Nogueira_2021}.~The
original product is provided at 0.05\textbf{°} spatial resolution with a
10-daily timestep, and it is aggregated to monthly values at
0.25\textbf{°~}to match the ERA5L LAI.~{From the reanalysis side, the
prescribed LAI is obtained from the ERA5 monthly averaged data on single
levels.} It is obtained by combining the variables called~\emph{{Leaf}
area index, high vegetation} and~\emph{Leaf area index, low vegetation}
on a per-grid cell basis using the fractions of high and low vegetation
prescribed in the model, known respectively as~\emph{High vegetation
cover} and~\emph{Low vegetation cover}.~


\subsection{Land surface temperature}

The observational reference for LST is obtained from the Moderate
Resolution Imaging Spectroradiometer (MODIS) instrument on-board of the
Aqua satellite platform. MODIS-Aqua was selected as its overpass time is
approximately 13:30 local time, which would be close to the time of the
daily maximum temperature. The precise MODIS data product is labeled as
MYD11A1 collection 6~\citep{Wan_2015}, based on a split-window
algorithm, and provides data at 1km spatial resolution at a daily
frequency. The variable in ERA5-Land that we compare LST to is called
"\emph{skin temperature}" and is defined as the theoretical temperature
that is required to satisfy the surface energy balance. We select skin
temperature at 14:00 so as to match as close as possible the overpass
time of the MODIS-Aqua instrument.~

To match the reanalysis variable with remote sensing observations,
special care is needed to address the clear-sky bias. The type of
thermal satellite data that we use can only provide an information on
the temperature's surface in the absence of clouds, which typically
leads to sampling the warmer days benefiting from unobstructed solar
radiation. The reanalysis dataset contains values for both sunny and
overcasts days, when the skin temperature is closer to air temperature.
To ensure comparability and have information at monthly scale \DIFdelbegin \DIFdel{, }\DIFdelend only the
5 warmest days of each month are selected from both the MYD11A1 and the
ERA5-Land datasets. \DIFdelbegin \DIFdel{~ }\DIFdelend To facilitate processing for the satellite data,
this procedure is directly implemented in the Google Earth Engine (GEE)
platform~\citep{Gorelick_2017}, which hosts a copy of the MYD11A1
catalogue. The aggregation to the 0.25\textbf{°} grid is done in a
second step. As a consequence of this matching procedure, the LST bias
is always referring to a bias in the five warmest \DIFdelbegin %DIFDELCMD < { %%%
\DIFdelend days of the month\DIFdelbegin %DIFDELCMD < }%%%
\DIFdelend .~
This assumes that the five warmest days are clear sky \DIFdelbegin \DIFdel{, but this
assumption should generally hold }\DIFdelend \DIFaddbegin \DIFadd{in both the satellite 
and the reanalysis, even if they might not be exactly the same days. Such 
assumption should hold in general for most conditions, especially }\DIFaddend as we are 
using LST at \DIFdelbegin \DIFdel{around }\DIFdelend 14:00, \DIFdelbegin \DIFdel{which is }\DIFdelend a variable that is highly sensitive to radiation. 
\DIFaddbegin \DIFadd{However, under some circumstances, such as in snowy conditions during 
wintertime, cloudy days may be warmer than clear-sky days.
}\DIFaddend 


\subsection{Albedo}

While the focus of this work is on the relationship between the LAI and
LST biases, it is also useful to investigate how biases in relevant
biophysical variables could help the mechanistic interpretation of the
discrepancies between LAI and LST in the modelling framework. The first
of these variables is albedo, the proportion of the incident solar
radiation that is reflected by the surface. It can serve as an indicator
of whether the LST bias in ERA5L is caused by the radiative effect of
changes in LAI, as an increase in LAI reduces albedo, which in turn
increases net radiation leading to a radiative warming of the surface.

The variable we use in ERA5 is "\emph{UV visible albedo for diffuse radiation}", 
which is the fraction of diffuse solar (shortwave) radiation with wavelengths between 
0.3 and 0.7 \selectlanguage{greek}µ\selectlanguage{english}m reflected by snow-free 
land surfaces. Like LAI, \DIFaddbegin \DIFadd{the snow-free }\DIFaddend albedo in ERA5 is not dynamic, but consists instead of a static seasonal climatology \DIFdelbegin \DIFdel{. }\DIFdelend \DIFaddbegin \DIFadd{based on satellite observations. We can note, however, that the changes in albedo due to snow are “prognostic”, i.e. it changes along with snow cover as modelled in the re-analysis system (i.e. in-sync with dynamic weather).  
}\DIFaddend From the satellite-based perspective, we use the 
standard MODIS daily albedo product, MCD43C3 V006~\citep{Schaaf_2015}. From this
dataset, we use the broadband white-sky estimations for the visible part of the spectrum, 
defined for this product as ranging from 0.4 to 0.7 
\selectlanguage{greek}µ\selectlanguage{english}m. 
This means some slight discrepancy with ERA5 might occur for
ultraviolet light over the 0.3 to 0.4 \selectlanguage{greek}µ\selectlanguage{english}m range, 
but this is expected to be very marginal due to the low contribution of UV light to ecosystem
scale albedo. There is a second discrepancy with ERA5 in that this MODIS albedo product is not snow-free, but this should not be a problem as albedo is only used here in the context of 
studying summer heatwaves were only minimal snow cover is expected over the mountain ranges.


\subsection{Land evaporation}

The second associated variable is land evaporation. This is the amount
of water that is evaporated from the land surface, including the
transpiration from vegetation,~ and which is transformed into water
vapour in the air above. In ERA5-Land, the variable is called
"\emph{Total evaporation}"~ and is provided as meters of water
equivalent at a monthly basis. Evaporation or transpiration cannot be
directly measured from satellite observations, but a data-driven
estimation can be obtained from dedicated modelling frameworks. The one
we use here is the Global Land Evaporation Amsterdam Model (GLEAM)
product~\citep{Martens_2017,Miralles_2011}.~ The GLEAM model estimates land evaporation
and its components: transpiration, bare-soil evaporation, interception
loss, open-water evaporation and sublimation. \DIFdelbegin \DIFdel{~ }\DIFdelend The version used here is
version 3.3b, which does not rely on ERA5 reanalysis data in order to
avoid any circularity in our benchmarking work. \DIFaddbegin \DIFadd{Similarly, while GLEAM uses various remote sensing products as inputs, LST is not one of them~\mbox{%DIFAUXCMD
\citep{Martens_2017} }\hskip0pt%DIFAUXCMD
thereby ensuring it can be confronted against MODIS LST without fearing circularity.
}\DIFaddend 


\subsection{Climate zones}

As land-atmosphere interactions are often related to the background
climate regime, it is often useful to stratify their analysis along with
some kind of climate zonation. Here we employ two different climate
classification approaches. The first consists in using the well-defined
K\selectlanguage{ngerman}öppen-Geiger classification scheme, as implemented for the period
1986-2010 by~\citep{Kottek_2006}.~The scheme defines five broad groups:
equatorial, arid, temperate, continental and polar, as well as subgroups
depending on the seasonal rainfall and temperature. The maps are
aggregated from their native spatial resolution of 1 km to the
0.25\textbf{°} grid using the nearest neighbour interpolation. To have
a finer division of climate along continuous axes of temperature and
aridity, a second climate zonation is done based on a division of the
world using intervals of yearly-averaged 2m air temperature and
yearly-averaged soil moisture. For this general purpose of climate
characterisation, these variables are collected from the monthly
ERA5-Land single layer dataset.~


\subsection{Heatwaves}

To isolate the specific effect of the interplay between the LAI and LST
biases during extreme events, this study also looks at three major
summer European heatwaves that have occurred in the recent past. All
three are characterised by a long duration and large-scale extent, but
varied in terms of geographic distribution and biomes affected. The
first is the heatwave of summer 2003 that hit western Europe, and
particularly France, and which will be here referred to as HW03. The
second is the Russian heatwave of July 2010, referred to henceforth as
HW10. The third heatwave considered occurred in 2018 and can be divided
into two zones where the effects had marked differences notably due to
contrasting land cover and background climate. The first zone we
consider is labelled HW18a and covers Northern Germany and Denmark, a
region dominated by croplands, while the second is located mostly over
forests in Finland and is labelled HW18b. The spatial extents of the
zones considered are represented in
Fig.~{\ref{946697}}, overlaid over the LST anomalies
from the period 2003-2018 for the~respective months considered for each
event. On this point, we underline the fact that the characterisation of
the heatwaves is done based on monthly data to remain consistent with
the rest of the analyses, despite the fact that heatwaves would be more
precisely defined by considering their duration more precisely at a
daily scale.

\selectlanguage{english}
\begin{figure}[H]
\begin{center}
\includegraphics[width=1.00\columnwidth]{../../figs/heatwave_intro}
\caption{{Delimitations of the areas considered for the various heatwave events
considered in this study. The LST anomalies presented are based on
satellite retrievals from MODIS.
{\label{946697}}%
}}
\end{center}
\end{figure}



\section{Results}


The outcomes of this study are all based on the analysis of biases
between ERA5L and other observational datasets for the specific
variables of LAI and LST. The results are structured along the two main
objectives mentioned before. The first part thus characterises the
general behaviour of how these biases interact based on their
climatologies, here considered as their mean cycles over the period
2003-2018. The second part then takes a more specific look on how these
biases interact during years that are different from this inter-annual
mean cycle, and more particularly on how this affects years of
heatwaves.

\par\null

\subsection*{Part 1: Characterisation of the patterns based on the
climatology}

{\label{835449}}

To begin, we first start with a general overview of how the biases in
LAI and LST are structured in space and time. The maps in
Fig.~{\ref{220254}} are composite images where winter
is represented by values for January in the Northern Hemisphere and
values for July in the Southern Hemisphere, while the reverse is true
for summer.~ The LAI in ERA5L is almost systematically higher than the
reference GEOV2-AVHRR LAI during winter, and this corresponds to an
overestimation of LST by ERA5L in the northern latitudes.~ This
relationship between the bias in LST and the bias in LAI is consistent
for such energy limited situation where biophysical effects of
vegetation on climate are dominated by radiative effects. \DIFdelbegin \DIFdel{In fact, }\DIFdelend \DIFaddbegin \DIFadd{As }\DIFaddend the
modelling framework assumes there is an excess of leaves covering the
background, \DIFdelbegin \DIFdel{the former being generally darker than the latter
(especially when the background is covered with snow ) resulting in more
}\DIFdelend \DIFaddbegin \DIFadd{and that this background is very likely covered by snow which is 
brighter than the simulated leaves, this may explain the higher
}\DIFaddend heat accumulation than what would be observed in a situation with fewer
leaves. Since the winter evapotranspiration is strongly limited by the
atmospheric evaporative demand at such high latitudes,~{there is no
compensation of the enhanced radiative warming from evaporative
cooling~\citep{bright2017local}. \DIFdelbegin \DIFdel{On the other hand, there is a considerable
underestimation of the LST for }\DIFdelend \DIFaddbegin \DIFadd{However, as mentioned previously 
care may be warranted when analysing the situation in winter in boreal areas. 
The pragmatic approach we have employed to match 
clear-sky LST days in the satellite record with those of the reanalysis 
(i.e. taking the warmest days) does not ensure we are strictly comparing the 
same days during wintertime, as winter days with snow and clear-sky conditions 
could be colder than overcast days because of a stronger radiative cooling. 
Another point regarding winter is that for the }\DIFaddend many drier parts of the world where
winter evapotranspiration is not energy-limited\DIFaddbegin \DIFadd{, there is a considerable
underestimation of the LST}\DIFaddend . This pattern can be explained by the additional 
evaporative cooling generated by the excess LAI in \DIFdelbegin \DIFdel{these }\DIFdelend \DIFaddbegin \DIFadd{such }\DIFaddend climate conditions.
\DIFdelbegin \DIFdel{~The situation is }\DIFdelend \DIFaddbegin 

\DIFadd{The situation in }\DIFaddend summertime is more complex, as LAI can either~be overestimated 
or underestimated~ depending on the geographical location. The biases in LST 
generally consist of an underestimation in ERA5L with respect to the satellite 
reference, which is particularly strong in deserts and drylands. There is a 
notable exception in Central and Western Africa where the LST is rather
overestimated in ERA5L. The interpretation of how both LAI and LST
biases are related is not straightforward due to the dynamic nature of
this relationship along the growing season and between climate regions.~

\par\null\selectlanguage{english}
\begin{figure}[H]
\begin{center}
\includegraphics[width=1.00\columnwidth]{../../figs/bias_summaries}
\caption{{{Overview of the mean biases} in LAI~and LST between ERA5L and
observations (ERA - obs) over the climatological period from 2003 to
2018. The panels represent composite maps for which the seasonalities of
both Northern and Southern Hemispheres are aligned: winter maps consist
of data for January in northern latitudes and July in southern
latitudes, while summer maps combine July values in the North with
January values in the South.
{\label{220254}}%
}}
\end{center}
\end{figure}

\par\null

To better diagnose the relationship between the bias in LST and the bias
in LAI, we plot them once against the other to analyse their cyclic
seasonal patterns as shown in Fig~{\ref{842529}}. These
plots summarise the bias for all areas under a specific climatic regime
defined by a range in a yearly average of soil moisture and 2m air
temperature. Fig~{\ref{842529}}a represents a region in
a tropical semi-humid climate,~ while Fig~{\ref{842529}}b is a colder and 
more humid climate. In both plots, a typical hysteretic pattern emerges, 
indicating how the relationship between the biases is depending on the 
background climate and changes during the course of the year. For the 
tropical region (Fig~{\ref{842529}}a), the relationship is consistent
with a land biophysical signal dominated by evaporative cooling: an
overestimation in LAI is associated with an underestimation in LST,
which is more pronounced in the months of January to May, while during
the year we observe a loop with smaller biases in summer than in fall.
For the second region (Fig~{\ref{842529}}b), the
growing season follows a stronger hysteretic pattern that even leads to
an underestimation of LAI by ERA5L (as had been observed in
Fig.~{\ref{220254}}), but there is a stark difference
in pattern for the wintertime where a strong change in LST bias occurs
independently of the bias in LAI. This pattern \DIFdelbegin \DIFdel{is }\DIFdelend \DIFaddbegin \DIFadd{can be }\DIFaddend consistent with the
explanation provided before in which a winter overestimation of LAI by
ERA5L in cold regions could lead to radiative warming due to the darker
surface of dense vegetation, which is not compensated \DIFdelbegin \DIFdel{~by ~}\DIFdelend \DIFaddbegin \DIFadd{by }\DIFaddend any additional
evaporative cooling due to the energy limitation of evapotranspiration
in winter conditions. \DIFdelbegin \DIFdel{~
}\DIFdelend \DIFaddbegin \DIFadd{However, it may also be affected by the potential 
wintertime mismatch of the 5 warmest days between ERA5L and the satellite 
data mentioned before.
}\DIFaddend 


\par\null\selectlanguage{english}
\begin{figure}[H]
\begin{center}
\includegraphics[width=1.00\columnwidth]{../../figs/hysteresis_demo}
\caption{{Diagnostic plots illustrating the hysteretic behaviour between the
biases in LAI and the biases in LST for two different climate zones. The
biases are always defined as ERA5L variables minus the values from
reference observational datasets. The small individual points represent
monthly values within the entire period 2003-2018. The larger points
represent the inter-annual mean values for each month (and the error
bars represent one standard deviation). The continuous line is obtained
from a harmonic fit.
{\label{842529}}%
}}
\end{center}
\end{figure}


\DIFdelbegin %DIFDELCMD < {%%%
\DIFdel{Fig~}%DIFDELCMD < }{%%%
\DIFdel{\ref{925458}}%DIFDELCMD < }{ }%%%
\DIFdel{further extends this analysis }\DIFdelend %DIF > \selectlanguage{english}
%DIF > \begin{figure}[h]
%DIF > \begin{center}
%DIF > \includegraphics[width=0.80\columnwidth]{../../figs/hystereris_climspace}
%DIF > \caption{{Hysteretic behaviour between the biases in LAI and the biases in LST for
%DIF > a range of climate zones defined by classes of annual mean 2m
%DIF > temperatures (T2m) and soil moisture (SM). The biases are always defined
%DIF > as ERA5L variables minus the values from reference observational
%DIF > datasets. Only interpolated curves are shown for clarity purposes.
%DIF > {\label{925458}}%
%DIF > }}
%DIF > \end{center}
%DIF > \end{figure}
%DIF > 
%DIF > \par\null
\DIFaddbegin 

\DIFadd{Such analysis can further be extended }\DIFaddend to a full climate space delimited by mean surface soil moisture and mean air temperature. \DIFdelbegin \DIFdel{This figure exposes }\DIFdelend \DIFaddbegin \DIFadd{A figure depicting the resulting effects of such gradients on the shapes of the hysteretic curves can be found in the supplementary material for this study. Such figure reveals }\DIFaddend two general gradients in the
patterns of how the LAI and LST biases behave during the seasonal cycle.
The first gradient shows how the hysteretic behaviour increases for
colder and moister climates, while very dry and hot regions show little
variation in either bias and thereby does~not show any significant
hysteretic patterns. \DIFdelbegin %DIFDELCMD < {%%%
\DIFdelend Beyond this first general gradient on the
intensity of \DIFdelbegin \DIFdel{hyste}%DIFDELCMD < }%%%
\DIFdel{resis}\DIFdelend \DIFaddbegin \DIFadd{hysteresis}\DIFaddend , there is second gradient showing a notable
difference between cold and dry regions, where the magnitude of the
seasonal variation in LST bias dominates, and warm and humid regions,
where this seasonal variation of the bias is stronger for LAI. 

\DIFdelbegin %DIFDELCMD < \selectlanguage{english}
%DIFDELCMD < \begin{figure}[h]
%DIFDELCMD < \begin{center}
%DIFDELCMD < \includegraphics[width=0.80\columnwidth]{../../figs/hystereris_climspace}
%DIFDELCMD < %%%
%DIFDELCMD < \caption{%
{%DIFAUXCMD
%DIFDELCMD < {%%%
\DIFdelFL{Hysteretic behaviour between the biases in LAI and the biases in LST for
a range of climate zones defined by classes of annual mean 2m
temperatures (T2m) and soil moisture (SM). The biases are always defined
as ERA5L variables minus the values from reference observational
datasets. Only interpolated curves are shown for clarity purposes.
}%DIFDELCMD < {\label{925458}}%%%
%DIF < 
%DIFDELCMD < }%%%
}
%DIFAUXCMD
%DIFDELCMD < \end{center}
%DIFDELCMD < \end{figure}
%DIFDELCMD < 

%DIFDELCMD < \par\null
%DIFDELCMD < 

%DIFDELCMD < {%%%
\DIFdel{In order }%DIFDELCMD < } %%%
\DIFdel{to quantify the two main gradients described in
Fig.~}%DIFDELCMD < {%%%
\DIFdel{\ref{925458}}%DIFDELCMD < }%%%
\DIFdelend \DIFaddbegin \DIFadd{In order to quantify these two gradients of hyteretic patterns}\DIFaddend , we propose 
two indices that respectively generalise these patterns of hysteretic intensity and bias
dominance. {{Hysteretic intensity (}}\emph{{HI}}{)} is simply summarised
by the total area (\emph{A}) formed by the hysteretic loop in a given
climate zone (\emph{i})\emph{~}divided by the area\emph{~}of the largest
loop encountered among all climate spaces:

\(HI\ =\ \frac{A_{i\ }}{\max\left(A\right)}\)

The area is calculated based on the smoothed seasonal cycles, which
themselves were fitted using \DIFdelbegin %DIFDELCMD < {%%%
\DIFdelend 3rd order harmonics fits applied
\DIFdelbegin \DIFdel{separetely }\DIFdelend \DIFaddbegin \DIFadd{separately }\DIFaddend to both variables as a function of time\DIFdelbegin %DIFDELCMD < }%%%
\DIFdel{. ~ }\DIFdelend \DIFaddbegin \DIFadd{. }\DIFaddend The area A is
calculated for each climate zone considering all intersecting loops as
generating positive areas, which would not be the standard procedure
from a topological perspective (as intersecting loops would generate
areas with opposite signs).

The second gradient \DIFdelbegin \DIFdel{in Fig.~}%DIFDELCMD < {%%%
\DIFdel{\ref{925458}}%DIFDELCMD < } %%%
\DIFdelend relates to describing which of the two biases (LAI or LST) 
dominates in terms of seasonal amplitude. The index to describe this behaviour 
follows the logic of a normalised difference index based on the standardised ranges
of both LAI and LST axes in the smoothed hysteresis curves. The
resulting bias dominance (\emph{BD}) index is expressed as follows:

\(BD\ =\ \frac{\left(\frac{range\left(x\right)}{\sigma_x}\ -\ 
\frac{range\left(y\right)}{\sigma_y}\right)}{\left(\frac{range\left(x\right)}{\sigma_x}\ 
+\ 
\frac{range\left(y\right)}{\sigma_y}\right)}\)

where~\emph{x~}stands for the bias in LAI and \emph{y} is the bias in
LST.

{These two indices can be mapped in climate space, but then also back
into geographical space, as shown in~
Fig.~}{\ref{988186}}{.} This provides a valuable
diagnostic that enables spatialisation of the magnitude of the
hysteretic discrepancies between ERA5L and observations in terms of the
interrelationship between their LAI and LST biases. This in turn is
useful both for users of reanalysis data, to know where the LAI/LST land
atmosphere interactions are to be expected to be problematic, and for
model developers, to know where they should prioritise model
improvements.~{{More specifically, when the HI map in
Fig.~}}{\ref{988186}}{ indicates a dark area, one knows
the relationship between biases does not have a strong seasonal
component}, and can instead be considered stable. In some cases, this is
because they resume to a single point (e.g. more desertic areas). In
others, it is because there is a clean quasi-linear relationship between
the LAI and the LST bias (e.g. tropical forests, or the example in
Fig.~{\ref{842529}}a), which could also be empirically
``corrected'' using a linear fit if this was deemed appropriate or
necessary for a user (although this would compromise the physical
integrity of the relationship between the variables in ERA5L).~ Areas
with high HI indicate there is a strong seasonal component in the
mismatch between LAI and LST biases, and this appears to affect areas
with strong seasonality in LAI and LST.

\par\null\selectlanguage{english}
\begin{figure}[H]
\begin{center}
\includegraphics[width=1.00\columnwidth]{../../figs/hystereris_maps}
\caption{{{Summary of how biases in LAI and LST interact} differently across
climate zones (left) and how these are translated back into geographic
space (right). {The HI index (top) indicates how important the
hysteretic patterns are. The BD index (bottom) indicates which of the
two bias dominates (between LAI and LST).}~
{\label{988186}}%
}}
\end{center}
\end{figure}

\par\null\par\null

\subsection*{Part 2: interannual variability and
heatwaves}

{\label{296419}}

After characterising the general patterns of the biases based on the
mean inter-annual cycles, or climatologies, we now turn the attention to
extreme situations which deviate from the mean. To begin, we start by
showing how the relationships between LST and LAI biases change from one
year to the next. This \DIFdelbegin %DIFDELCMD < {%%%
\DIFdelend is done via an analysis of \DIFdelbegin %DIFDELCMD < } %%%
\DIFdelend their inter-annual
variability for a specific month and place over the considered period
(from~{2003} to 2018). Fig.~{\ref{891702}} displays the
temporal correlation between the biases for selected months representing
the seasons uniformly across the world (i.e. seasonality of the Northern
and Southern Hemispheres are aligned). The most prominent patterns are
negative correlations in drier areas, especially when there is strong
radiation load in summer. This comforts the previous assessment that an
overestimation of LAI in the modelling framework coincides with an
underestimation of the LST, but further indicates how this effect
changes on a year-to-year basis. In other words, the years when the
seasonally prescribed LAI of ERA5L is further from the reality, e.g. in
years where the LAI peak is lower or shifted due to particular \DIFdelbegin \DIFdel{~}\DIFdelend growing
conditions of that \DIFdelbegin \DIFdel{~}\DIFdelend year, the underestimation of LST can be expected to
be more severe. \DIFdelbegin \DIFdel{~
}\DIFdelend \DIFaddbegin \DIFadd{Care is warranted while interpreting these results, as 
LAI anomalies are also reflected on other aspects of the land surface 
(e.g. drier soils, changes in albedo, changes in surface roughness), but 
it remains that ERA5L seems to show larger errors during extremes.
}\DIFaddend 

\par\null\selectlanguage{english}
\begin{figure}[H]
\begin{center}
\DIFdelbeginFL %DIFDELCMD < \includegraphics[width=0.70\columnwidth]{../../figs/corr_summaries}
%DIFDELCMD < %%%
\DIFdelendFL \DIFaddbeginFL \includegraphics[width=0.60\columnwidth]{../../figs/corr_summaries}
\DIFaddendFL \caption{{Inter-annual correlation between the biases in LAI and the biases in LST
based on all months of July and January over the period 2003-2018. As
with Fig.~{\ref{220254}} these are composite maps for
which the seasonalities of both Northern and Southern Hemispheres are
aligned: winter maps consist of data for January in northern latitudes
and July in southern latitudes, while summer maps combine July values in
the North with January values in the South. \DIFdelbeginFL \DIFdelFL{~
}\DIFdelendFL \DIFaddbeginFL \DIFaddFL{Correlations are only shown if based on more than 10 years and when deemed statistically significant (p-value < 0.05).
}\DIFaddendFL {\label{891702}}%
}}
\end{center}
\end{figure}

To better understand how the relationship between LAI and LST reacts
under conditions that deviate from the normal, the next analysis
concentrates on using anomalies of temperature as a grouping variable.
For the scope of this analysis, the focus is placed on Europe \DIFaddbegin \DIFadd{and for the month of August}\DIFaddend .
Fig.~{\ref{560364}} summarises how the LAI and LST
biases evolve when considering the full range of temperature anomalies
encountered in our dataset across different climate zones in Europe. \DIFdelbegin %DIFDELCMD < {%%%
\DIFdelend To
construct this plot\DIFdelbegin %DIFDELCMD < }%%%
\DIFdelend , the entire distribution of values for a given bias
are considered \DIFdelbegin \DIFdel{, effectively mixing space and time}\DIFdelend \DIFaddbegin \DIFadd{for the month of August}\DIFaddend . This distribution is
divided in quantiles (deciles in this case) based on their value of land
surface temperature monthly anomalies. For each group of anomalies, the
average bias in LST or LAI is shown. There is a clear difference across
climate zones. For LAI, the bias is relatively stable irrespective of
LST anomalies in subartic climate (Dfc), but it has a tendency to
increase with higher LST anomalies in humid continental climate (Dfb) or
in oceanic climate (Cfb), while in mediterranean climate (Csa) it
actually decreases when extremes occur. LST largely follow the opposite
patterns for the warm extremes, but not necessarily for the cold ones
(most notably in Dfc and Cfb).~ For the specific case of heatwaves,
Fig.~{\ref{560364}} suggests that for most of
continental Europe, the bias in LAI will go from an underestimation in
ERA5L to an overestimation as thermal anomalies increase, and that this
will considerably aggravate the discrepancies in LST.

\selectlanguage{english}
\begin{figure}[H]
\begin{center}
\includegraphics[width=1.00\columnwidth]{../../figs/bias_vs_anomaly}
\caption{{Description of how the biases in both LAI and LST (between ERA5L and
observations) changes over different climate zones within Europe
depending on anomaly intensities in LST \DIFaddbeginFL \DIFaddFL{for the month of August}\DIFaddendFL .
{\label{560364}}%
}}
\end{center}
\end{figure}

Finally, we turn out attention to the specific case of the three major
European heatwaves in 2003, 2010 and 2018 with the latter one divided
among the two regions (HW18a and HW18b).
Fig.~{\ref{116908}} maps the differences in the biases
between the year of the heatwave and the average bias for the same
period, and what we refer to here as a bias shift. This bias shift only
informs us on how the bias changes from the normal year to a heatwave
year, but does not indicate whether the~ starting situation is an
overestimation or an underestimation. Therefore, to facilitate the
interpretation,~ Fig.~{\ref{116908}} also includes the
spatially averaged biases for each event under the maps.~

The first point to remark in Fig.~{\ref{116908}} is
that for HW03, HW10 and HW18a~ there is a considerable LAI bias shift in
the same direction due to the fact that LAI is effectively lower in the
observed dataset during these events than in normal years. For HW03 and
HW10, this changes the situation from an underestimation by ERA5L in
normal years to a strong overestimation during heatwave conditions. For
HW18 the situation is different, both with respect to the other
heatwaves, but also among the two sub-regions~considered. For the HW18a
region over Northern Germany and Denmark, the LAI bias shift actually
leads to a situation in which the prescribed LAI in ERA5L is actually
closer to the reduced LAI measured during that specific year, leading
effectively to less underestimation than in normal situations. For the
HW18b region in Finland, which is dominated by forests unlike all other
considered regions, the LAI bias shift is in the opposite direction
going from an overestimation of LAI by ERA5L to a slight underestimation
in the heatwave year.

The second point to highlight in Fig.~{\ref{116908}} is
that all heatwave cases show a negative LST bias shift, albeit with
different orders of magnitude.~ For HW03 and HW10, the LST bias shift is
very strong, and it clearly corresponds to the positive shift in LAI
bias. This confirms that ERA5L can suffer from a cold bias in these
extreme situations, arguably attributable to an excessive evaporative
cooling caused by simulating many more leaves than what is present in
reality. In contrast, the spatially averaged shift in LST bias for HW18b
is almost insignificant, with even some increases in some areas. This is
in line with the remark that LAI over these forested areas may be better
estimated during this event by the ERA5L prescribed climatology,
resulting in very little consequences to the LST bias. In the case of
HW18a over northern Germany, the improvement in LAI for the event almost
entirely removes the positive LST bias that exists in normal conditions.

To better understand how the biases in LST and LAI are effectively
related in the contrasting heatwave circumstances,
Fig.~{\ref{116908}} also provides the same maps for the
shift in two other variables:~ shortwave albedo and total evaporation.
The albedo maps show the shift that would be expected over
cropland-dominated areas during heatwaves, i.e. the senescence of
cereals would be accelerated resulting in brighter surfaces as cereals
dry off, resulting in a negative albedo shift when comparing the real
observed albedo with the prescribed one. This is clearly not visible
over the forested HW18b zone where the LAI bias present in normal years
is somewhat corrected during heatwave years. The evaporation bias shift
shows a different pattern. For HW03 and HW10, the heatwaves aggravate
the overestimation of evaporation, which is coherent with the excess
simulation of leaves in the model and the corresponding non-radiative
cooling that they would cause. The situation in HW18b also shows a
positive shift of the same order of magnitude, but in this case it goes
from a large underestimation of evaporation to a somewhat milder
underestimation, which is consistent with the fact that there is less of
a LAI bias. The likely explanation for this contrasting behaviour may
lie in the strength of the soil-moisture/temperature coupling, which is
high for HW03 and HW10 but less important for HW18b~\citep{liu2020similarities},
and this in turn depends on differences in land cover and background
climate. Croplands and grasslands dominating HW03 and HW10 deplete soil
moisture more readily than forests in HW18b, thereby triggering a more
rapid release of sensible fluxes, while forests can tolerate heatwaves
better thanks to deeper roots and the fact the in these northern
latitudes of HW18b, the soil moisture evaporation that is lower.~

The stark difference is HW18a, which one would assume would behave more
like HW10 and HW03,~ and that the overestimation of leaves by ERA5L
leads to more simulated evaporation which in turn leads to a colder
bias. Instead, the evaporation bias shifts in the other direction, going
from no underestimation~ to a strong underestimation, and yet a cooling
LST bias shift is also observed. This may be linked to uncertainties in
the GLEAM product, which is here considered as the reference
observations. GLEAM does not directly measure evaporation, but rather
infers it from the data based on several modelling assumptions. Compared
to flux tower estimations, GLEAM was also shown to underestimate
transpiration more than ERA5-Land~\citep{Munoz_Sabater_2021}. Therefore, 
the discrepancy in HW18a may require more investigation based on other
reference sources.

\par\null\selectlanguage{english}
\begin{figure}[H]
\begin{center}
\includegraphics[width=1.00\columnwidth]{../../figs/heatwaves}
\caption{{Maps of bias shifts for different variables when comparing ERA5L to what
is considered here as observations. The bias shift consists of
differences in the biases between the year of the heatwave and the
average bias for the same period. Below the maps, we show the actual
biases for each variable for the average climatological bars (light bar)
and for the year of the heatwave (dark bar). The colour of the bars
represents the direction of the bias shift.
{\label{116908}}%
}}
\end{center}
\end{figure}



\section{Discussion}

The present study proposes a novel diagnostic for land surface models
centred around the key variable of LST.~In the particular case of our
evaluation of ERA5L, the analysis reveals the magnitude of the LST bias
and its strong but heterogenous co-variation with spatio-temporal biases
in LAI. It further demonstrates that these have even stronger
consequences for heatwaves, when the bias in LST caused by the
mis-representation of LAI is often exacerbated. The main outcome \DIFdelbegin \DIFdel{therefore is }\DIFdelend \DIFaddbegin \DIFadd{of this 
study is therefore }\DIFaddend a general warning for users of both ERA5 and ERA5-Land 
\DIFdelbegin \DIFdel{that
the magnitude of heatwaves in particular may be underestimated in these datasets}\DIFdelend \DIFaddbegin \DIFadd{about the possible shortcomings these datasets may have under heatwave 
conditions. Furthermore, if heatwaves were to be defined based on the skin 
temperature using these datasets, their magnitude would be seriously 
underestimated}\DIFaddend . A secondary caveat is that these datasets should not be used
to assess the sensitivity of LAI to \DIFdelbegin \DIFdel{temperature}\DIFdelend \DIFaddbegin \DIFadd{skin temperature (i.e. LST)}\DIFaddend , nor 
to other variables related to the surface energy balance, as there is a 
clear disconnection between both.

LST is particularly suited to assess how models represent
land-atmosphere interaction as it summarises an equilibrium point of the
energy balance that can be easily observed from observations. Other
variables of the energy balance, such as the latent heat flux, cannot be
captured so directly by observations, requiring instead several
modelling steps along with their associated assumptions. Large
discrepancies between observed and simulated LST are a strong indicator
that there is a problem regarding how energy is partitioned in the
model, which will further generate uncertainty in the representation of
the atmosphere. In our case, there was a known suspect for the problem:
the mis-representation of the inter-annual variability of LAI phenology
in the ERA5L setup. However, the diagnostic we propose could easily be
generalised to other state variables determining the energy
partitioning. The effect of the static representation of land cover in
ERA5L could be a first example. Although, for the purposes of this
study, we considered that the changes in LAI would implicitly
incorporate changes in land cover, there are further layers of subtlety
to be evaluated. Indeed, the distinction between different classes of
low and high vegetation~can have the same LAI but with different
clumping patterns, resulting in different roughness lengths, which
themselves could have different effects on LST. 

Several discussion points can be raised with regards to extending or
improving the present framework for model evaluation. A first aspect
relates to the clear-sky bias in the satellite remote sensing data. \DIFdelbegin \DIFdel{Our
}\DIFdelend \DIFaddbegin \DIFadd{As 
mentioned before, our }\DIFaddend approach to focus on the subset of days within the 
month that have the highest values in ERA5L should generally be robust to ensure
comparability with the highest values of LST measured by the satellites,
especially during the warmer season when clear skies are directly
associated with higher temperatures. This \DIFdelbegin \DIFdel{max }\DIFdelend \DIFaddbegin \DIFadd{maximum }\DIFaddend LST metric may be less
effective or appropriate in wintertime, as this assumption may not
always hold: clear-sky days may sometimes be colder than overcast days.
Because the measurements are done in the early afternoon when radiation
load is high, it is still reasonable to believe that radiation will be
the main driver determining skin temperature (rather than air
temperature) in many cases, but arguably the assumption may not be as
strong in winter \DIFdelbegin \DIFdel{than }\DIFdelend \DIFaddbegin \DIFadd{as }\DIFaddend in summer. A possible improvement could be to work
with daily values and select explicitly days in ERA5L that have
clear-sky conditions. Working at a daily scale would have the added
benefit of being able to isolate the effects of heatwaves more precisely
than with the monthly scale used here.~Replacing MODIS LST observations
with LST from geostationary satellite data, such as SEVERI on-board of
MSG, could even allow pushing further by sampling from different parts
of the diurnal cycle augmenting the chances of clear-sky observations.
However, in all cases there is still the complication that matching
clear-sky observations with the daily (or sub-daily) modelled clear-sky
simulations is hampered by the model's capacity to correctly model
clear-sky conditions, which could arguably be affected by the
misrepresentation of LAI, introducing some kind of circularity.

Another point where the current model diagnostic could be improved
relates to the associated variables used to \DIFdelbegin \DIFdel{~}\DIFdelend help interpret the
mechanistic interpretation of the discrepancies between LAI and LST. In
the present work, we limited ourselves to albedo from MODIS and
evaporation from GLEAM, and only for the heat wave analysis in the
summer. First, their use as explanatory variables could be extended
beyond the summer period. This was currently not done because the values
provided in ERA5L only represent snow-free albedo, which do not reflect
the same reality as the MODIS albedo covering all conditions. \DIFaddbegin \DIFadd{A possible 
improvement could be to compute a MODIS comparable albedo from other 
variables that are available within ERA5L (i.e. from the surface net solar 
radiation and the downwards surface solar radiation). }\DIFaddend Second,
the GLEAM v3.3b used here relies on vegetation optical depth (VOD) to
characterise vegetation growth, a variable that is sensitive to humidity
conditions and which may thus not always be comparable with the LAI
signal estimated from optical instruments. This may partially explain
the inconsistencies in what is happening in HW18a, as the regions of
Northern Germany and Denmark that witnessed that specific event are more
humid in general than the areas in France and Russia where the other
heat waves occurred. Third, other types of such diagnostic variables
could be used. A prime candidate could be soil moisture itself (SM),
estimated from microwave remote sensing. In our case, we declined from
using it because the corresponding C3S project had many spatial gaps
(especially for year 2003) that complicated their interpretation when
comparing it to the other variables.

\DIFaddbegin \DIFadd{The use of LAI in both ERA5 and ERA5-Land deserves a little 
more discussion. Currently, LAI is not directly used in the land surface 
model, but it is rather used as a predictor for certain parameters in 
some parametrisations, these latter replacing some processes that are 
too small or complex to be physically represented and explicitly resolved. 
In the case of processes relevant towards representing LST for instance, the 
canopy resistance is parametrised based on LAI within ERA5 and ERA5-Land. 
The present study can advance our knowledge of the modelling system by 
clearly showing the relation between vegetation status 
(via the LAI) and the LST biases, suggesting it can be improved both by revising 
the actual LAI data that is used, but also how the model uses this data in 
the different parameterisations. Furthermore, the LAI data could also be 
used dynamically beyond static parameterisations by incorporating it in 
under a data assimilation scheme. There are pragmatic reasons why LAI is 
currently not being used dynamically within the ERA5 modelling framework. Some 
of ERA5's main strengths are its consistency and temporal depth, with an archive 
going back to the 1940s. To be properly assimilated, LAI should be available 
throughout the entire period, while the satellite era does not reach that far.
}

\DIFaddend Despite the strong discrepancies in terms of LAI and LST biases that we
present in this study, it is important to point out that~ERA5-Land and
ERA5 remain invaluable assets for the field. They remain for many the
best tools to describe many meteorological state variables in a
consistent way at hourly scale since the \DIFdelbegin \DIFdel{1950s}\DIFdelend \DIFaddbegin \DIFadd{1940s}\DIFaddend . We certainly continue to
encourage their use. However, we should stress that the biases we expose
in our results indicate that some diagnostics based on the relationships
between variables in ERA5-Land and ERA5, such as assessing the
sensitivity of LAI to temperature, should probably not be done as it
could lead to wrong assessments. \DIFaddbegin \DIFadd{Finally, another point to raise is that 
the LST biases in ERA5L are not completely explained by LAI. Other factors 
can also come into play, such as the absence of a proper representation of 
irrigation, misrepresentation of snow, altitude, slope effects in complex 
terrain and solar radiation biases in mountainous area. Improvements in 
these fields could also translate in a reduction of biases and should be pursued.
}\DIFaddend 

\conclusions  %% \conclusions[modified heading if necessary]

The present work provides a new perspective on the importance for land
surface modelling schemes to capture the dynamical nature of the
interface between vegetation and atmosphere. Basically, getting the
leaves right matter. Biases in LAI, which integrate this relationship
between surface and atmosphere,~are shown to be strongly correlated to
discrepancies in the representation of surface temperature within the
modelling framework behind the highly used ERA5 and ERA5-Land
meteorological reanalysis datasets. The impact of not
simulating~dynamically the LAI cycle is more acutely demonstrated by
focusing the particular case of heat waves in Europe, where we show how
their magnitude in terms of LST may be considerably underestimated. By
characterising and mapping the interplay between these LAI and LST
biases, our work may help users of these reanalysis datasets to
anticipate where and when larger uncertainties could be expected.~ It
should also help model developers to improve their current modelling
setups by establishing a performance benchmark, and by pinpointing where
and when the larger biases occur.~

Overall, ECMWF analyses and reanalyses will continues to pursue the
benefits of coupled data assimilation~\citep{de_Rosnay_2022}, but the
availability of stand-alone land analyses methods~\citep{Fairbairn_2019}
permit to examine the impact of assimilating LAI (and other land climate
data records datasets) to further reduce the LST biases in future
dedicated land reanalyses.~Ultimately, our work does provides a strong
argument to push for the assimilation of land surface variables that can
be measured from satellite Earth observation, such as LAI and LST, in
the weather forecasting system of ECMWF.~Finally, in a more generic
conclusion reaching beyond the ECMWF system, this study could provide a
blueprint for a wider benchmarking framework for land surface model
evaluation that exploits the capacity of LST to integrate effects of
both radiative and non-radiative processes affecting the surface energy
balance.

%DIF < % The following commands are for the statements about the availability of data sets and/or software code corresponding to the manuscript.
%DIF < % It is strongly recommended to make use of these sections in case data sets and/or software code have been part of your research the article is based on.
%DIF > % The following commands are for the statements about the availability of data sets and/or 
%DIF > % software 
%DIF > % code corresponding to the manuscript.
%DIF > % It is strongly recommended to make use of these sections in case data sets and/or software 
%DIF > % code 
%DIF > % have been part of your research the article is based on.

%\codeavailability{TEXT} %% use this section when having only software code available


%\dataavailability{TEXT} %% use this section when having only data sets available


\codedataavailability{The code necessary to reproduce this analysis is available in a Zenodo 
repository: https://doi.org/10.5281/zenodo.7275088. The input data for this work is available in 
another dedicated Zenodo repository: https://doi.org/10.5281/zenodo.6976942} %DIF < % use this section when having data sets and software code available
%DIF > % use this section 
%DIF > % when 
%DIF > % having data sets and software code available


%\sampleavailability{TEXT} %% use this section when having geoscientific samples available


%\videosupplement{TEXT} %% use this section when having video supplements available


%\appendix
%\section{}    %% Appendix A

%\subsection{}     %% Appendix A1, A2, etc.


%DIF < \noappendix       %% use this to mark the end of the appendix section. Otherwise the figures might be numbered incorrectly (e.g. 10 instead of 1).
%DIF > \noappendix       %% use this to mark the end of the appendix section. Otherwise the figures 
%DIF > might 
%DIF > be numbered incorrectly (e.g. 10 instead of 1).

%DIF < % Regarding figures and tables in appendices, the following two options are possible depending on your general handling of figures and tables in the manuscript environment:
%DIF > % Regarding figures and tables in appendices, the following two options are possible depending 
%DIF > % on your general handling of figures and tables in the manuscript environment:

%DIF < % Option 1: If you sorted all figures and tables into the sections of the text, please also sort the appendix figures and appendix tables into the respective appendix sections.
%DIF > % Option 1: If you sorted all figures and tables into the sections of the text, please also 
%DIF > % sort 
%DIF > % the appendix figures and appendix tables into the respective appendix sections.
%% They will be correctly named automatically.

%DIF < % Option 2: If you put all figures after the reference list, please insert appendix tables and figures after the normal tables and figures.
%DIF > % Option 2: If you put all figures after the reference list, please insert appendix tables and 
%DIF > % figures after the normal tables and figures.
%% To rename them correctly to A1, A2, etc., please add the following commands in front of them:

\appendixfigures  %% needs to be added in front of appendix figures

\appendixtables   %% needs to be added in front of appendix tables

%DIF < % Please add \clearpage between each table and/or figure. Further guidelines on figures and tables can be found below.
%DIF > % Please add \clearpage between each table and/or figure. Further guidelines on figures and 
%DIF > % tables 
%DIF > % can be found below.



\authorcontribution{GD, MP and AC designed the study. MP gathered and preprocessed the data.
GD and MP made the analyses and the figures. GD prepared the manuscript
with contributions from all co-authors.} %% this section is mandatory



\competinginterests{The authors declare that they have no conflict of interest.} %DIF < % this section is mandatory even if you declare that no competing interests are present
%DIF > % this section 
%DIF > % is mandatory even if you declare that no competing interests are present

%\disclaimer{TEXT} %% optional section

\begin{acknowledgements}

Gregory Duveiller acknowledges funding by the European Research Council
(ERC) Synergy Grant ``Understanding and modeling the Earth System with
Machine Learning (USMILE)'' under the Horizon 2020 research and
innovation programme (Grant agreement No. 855187). The GEOV2/AVHRR LAI 
product was generated by CNES in the framework of
the Theia land data centre, a French national inter-agency organization.
The GEOV2/AVHRR algorithm was developed by CREAF and INRAE. The research
leading to the current version of the product has received initial
funding from various European Commission Research and Technical
Development programs. The product is based on AVHRR 1km data ((c) NOAA)
and is distributed by Theia.

%GEOV2-AVHRR was produced and distributed by CNES based on the algorithm
%developed by CREAF and INRAE in the framework of the Theia Land Data
%Centre.

\end{acknowledgements}


\bibliographystyle{copernicus}
\bibliography{biblio.bib}

%% REFERENCES

%% The reference list is compiled as follows:

%% \begin{thebibliography}{}

%% \bibitem[AUTHOR(YEAR)]{LABEL1}
%% REFERENCE 1

%% \bibitem[AUTHOR(YEAR)]{LABEL2}
%% REFERENCE 2

%% \end{thebibliography}

%% Since the Copernicus LaTeX package includes the BibTeX style file copernicus.bst,
%% authors experienced with BibTeX only have to include the following two lines:
%%
%% \bibliographystyle{copernicus}
%% \bibliography{example.bib}
%%
%% URLs and DOIs can be entered in your BibTeX file as:
%%
%% URL = {http://www.xyz.org/~jones/idx_g.htm}
%% DOI = {10.5194/xyz}


%% LITERATURE CITATIONS
%%
%% command                        & example result
%% \citet{jones90}|               & Jones et al. (1990)
%% \citep{jones90}|               & (Jones et al., 1990)
%% \citep{jones90,jones93}|       & (Jones et al., 1990, 1993)
%% \citep[p.~32]{jones90}|        & (Jones et al., 1990, p.~32)
%% \citep[e.g.,][]{jones90}|      & (e.g., Jones et al., 1990)
%% \citep[e.g.,][p.~32]{jones90}| & (e.g., Jones et al., 1990, p.~32)
%% \citeauthor{jones90}|          & Jones et al.
%% \citeyear{jones90}|            & 1990



%% FIGURES

%DIF < % When figures and tables are placed at the end of the MS (article in one-column style), please add \clearpage
%DIF < % between bibliography and first table and/or figure as well as between each table and/or figure.
%DIF > % When figures and tables are placed at the end of the MS (article in one-column style), please 
%DIF > % add \clearpage
%DIF > % between bibliography and first table and/or figure as well as between each table and/or 
%DIF > % figure.

%DIF <  The figure files should be labelled correctly with Arabic numerals (e.g. fig01.jpg, fig02.png).
%DIF >  
%DIF > % The figure files should be labelled correctly with Arabic numerals (e.g. fig01.jpg, 
%DIF >  fig02.png).


%DIF < % ONE-COLUMN FIGURES
%DIF > % 
%DIF >  ONE-COLUMN FIGURES

%%f
%\begin{figure}[t]
%\includegraphics[width=8.3cm]{FILE NAME}
%\caption{TEXT}
%\end{figure}
%
%%% TWO-COLUMN FIGURES
%
%%f
%\begin{figure*}[t]
%\includegraphics[width=12cm]{FILE NAME}
%\caption{TEXT}
%\end{figure*}
%
%
%%% TABLES
%%%
%%% The different columns must be seperated with a & command and should
%%% end with \\ to identify the column brake.
%
%%% ONE-COLUMN TABLE
%
%%t
%\begin{table}[t]
%\caption{TEXT}
%\begin{tabular}{column = lcr}
%\tophline
%
%\middlehline
%
%\bottomhline
%\end{tabular}
%\belowtable{} % Table Footnotes
%\end{table}
%
%%% TWO-COLUMN TABLE
%
%%t
%\begin{table*}[t]
%\caption{TEXT}
%\begin{tabular}{column = lcr}
%\tophline
%
%\middlehline
%
%\bottomhline
%\end{tabular}
%\belowtable{} % Table Footnotes
%\end{table*}
%
%%% LANDSCAPE TABLE
%
%%t
%\begin{sidewaystable*}[t]
%\caption{TEXT}
%\begin{tabular}{column = lcr}
%\tophline
%
%\middlehline
%
%\bottomhline
%\end{tabular}
%\belowtable{} % Table Footnotes
%\end{sidewaystable*}
%
%
%%% MATHEMATICAL EXPRESSIONS
%
%%% All papers typeset by Copernicus Publications follow the math typesetting regulations
%%% given by the IUPAC Green Book (IUPAC: Quantities, Units and Symbols in Physical Chemistry,
%DIF < %% 2nd Edn., Blackwell Science, available at: http://old.iupac.org/publications/books/gbook/green_book_2ed.pdf, 1993).
%DIF > %% 2nd Edn., Blackwell Science, available at: 
%DIF > %% http://old.iupac.org/publications/books/gbook/green_book_2ed.pdf, 
%DIF > %% 1993).
%%%
%%% Physical quantities/variables are typeset in italic font (t for time, T for Temperature)
%%% Indices which are not defined are typeset in italic font (x, y, z, a, b, c)
%%% Items/objects which are defined are typeset in roman font (Car A, Car B)
%DIF < %% Descriptions/specifications which are defined by itself are typeset in roman font (abs, rel, ref, tot, net, ice)
%DIF > %% Descriptions/specifications which are defined by itself are typeset in roman font (abs, rel, 
%DIF > %% ref, tot, net, ice)
%%% Abbreviations from 2 letters are typeset in roman font (RH, LAI)
%%% Vectors are identified in bold italic font using \vec{x}
%%% Matrices are identified in bold roman font
%DIF < %% Multiplication signs are typeset using the LaTeX commands \times (for vector products, grids, and exponential notations) or \cdot
%DIF > %% Multiplication signs are typeset using the LaTeX commands \times (for vector products, 
%DIF > %% grids, 
%DIF > %% and exponential notations) or \cdot
%%% The character * should not be applied as mutliplication sign
%
%
%%% EQUATIONS
%
%%% Single-row equation
%
%\begin{equation}
%
%\end{equation}
%
%%% Multiline equation
%
%\begin{align}
%& 3 + 5 = 8\\
%& 3 + 5 = 8\\
%& 3 + 5 = 8
%\end{align}
%
%
%%% MATRICES
%
%\begin{matrix}
%x & y & z\\
%x & y & z\\
%x & y & z\\
%\end{matrix}
%
%
%%% ALGORITHM
%
%\begin{algorithm}
%\caption{...}
%\label{a1}
%\begin{algorithmic}
%...
%\end{algorithmic}
%\end{algorithm}
%
%
%%% CHEMICAL FORMULAS AND REACTIONS
%
%%% For formulas embedded in the text, please use \chem{}
%
%%% The reaction environment creates labels including the letter R, i.e. (R1), (R2), etc.
%
%\begin{reaction}
%%% \rightarrow should be used for normal (one-way) chemical reactions
%%% \rightleftharpoons should be used for equilibria
%%% \leftrightarrow should be used for resonance structures
%\end{reaction}
%
%
%%% PHYSICAL UNITS
%%%
%%% Please use \unit{} and apply the exponential notation


\end{document}
